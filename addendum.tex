%&pdflatex
\documentclass[12pt]{article}
\usepackage[british]{babel}

\usepackage{hyperref}
\usepackage{natbib}
\usepackage{tcolorbox}
\usepackage{booktabs,siunitx}

\usepackage{tikz-timing}

\usepackage[right=2cm, vscale=0.8, includemp, marginparsep=5mm,
marginparwidth=1.8cm]{geometry}

\title{KF6010 Mini Project --- Addendum}
\author{Dr Alun Moon\thanks{\url{alun.moon@northumbria.ac.uk}} and
Dr David Kendall\thanks{\url{david.kendall@northumbria.ac.uk}}}
\begin{document}
\maketitle
\appendix
\begin{appendix}
	Here are some design notes on my version of the pedestrian crossing.
\end{appendix}
\section{Threads/Tasks}
There are three threads running the tasks, within each task timings are handled
by simple \texttt{wait} calls. 
\paragraph{Vehicles} handles the light sequence for the vehicle lights.
\paragraph{Pedestrians} handles the light sequence for the pedestrians.
\paragraph{Bleep} handles the on-off mechanism for the audible signal.

\section{Control, Synchronisation, and Signalling}
\subsection{Mutex}
The crossing is a critical shared resource.
\paragraph{A Mutex} is used to control access to the shared piece of tarmac,
so only vehicles or pedestrians can use the crossing.

\subsection{Button signal}
The push button to signal a pedestrian is waiting.  
The thread handling the vehicle lights, waits on a semaphore while the light  is green.
\paragraph{Interrupt}  The interrupt handler for the button(s) releases the
semaphore waited on by the vehicle thread.
\paragraph{Semaphore} \texttt{tocross} signals that pedestrians are waiting.
(The vehicle thread uses \texttt{tocross.wait()}
\paragraph{Semaphore} \texttt{bleeping} is used to start and stop the bleeping
in synchronisation with the pedestrian crossing thread.
\section{Calculations}
\subsection{Frequency}
The tone can be generated with a PWM output.  The period for the PWM output is
calculated in table~\ref{tone-freq}

\begin{table}[h]
	\caption{Audible tone frequency}
	\label{tone-freq}
	\centering
\begin{tabular}{lSs}\toprule
	Frequency & 2 & \kilo\hertz \\
			  &  2000 & \hertz \\
	period	&	0.0005 & \second \\
			& 0.5	& \milli\second \\\midrule
PWM period	& 500 & \micro\second\\\bottomrule
\end{tabular}
\end{table}

\subsection{Pulse}
The pulse timing is calculated in tables~\ref{pulse-rate} and \ref{mark-space}
A mark-space ratio of 1.5:1 is 3:2\\

\begin{table}[h]
	\caption{Audible pulse timings}
	\label{pulse-rate}
	\centering
\begin{tabular}{lSs}\toprule
	Pulses & 240 & \per\minute \\
			& 4  & \hertz \\
	period & 0.25 & \second \\\midrule
		& 	250 & \milli\second\\\bottomrule
\end{tabular}
\end{table}


\begin{table}[h]
	\caption{Mark-space ratio}
	\label{mark-space}
	\centering
\begin{tabular}{lcSs}\toprule
	mark & $\frac{3}{5}$ period  & 150 & \milli\second \\
	space  & $\frac{2}{5}$ period  & 100 & \milli\second \\\bottomrule
\end{tabular}
\end{table}

\end{document}

