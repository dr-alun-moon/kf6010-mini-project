%&pdflatex
\documentclass{article}
\usepackage[british]{babel}

\usepackage[vscale=0.8, hscale=0.75, includemp, marginparwidth=1.5cm]{geometry}

\usepackage{hyperref}
\usepackage{todonotes}

\usepackage{tikz-timing}

\title{KF6010 Mini Project}
\author{Dr Alun Moon\thanks{\url{alun.moon@northumbria.ac.uk}} and
Dr David Kendall\thanks{\url{david.kendall@northumbria.ac.uk}}}
\begin{document}
\maketitle

\begin{abstract}
\end{abstract}

\section{Introduction}
	This mini project runs at the end of the first semester, and can be viewed
	as a formative exercise to get feedback ready for the final assessment at
	the end of the module.
	\begin{quote}
		Do treat this as seriously as if it were an assignment.
		Because this isn't assessed, we can give you much more help and
		feedback than we can with the final assignment.  

		\emph{Take this opportunity to build up your knowledge and confidence.}
	\end{quote}

\section{Scenario}
A pedestrian crossing has the following features:
\begin{itemize}
	\item Traffic lights to control traffic on the road
	\item Pedestrian lights to control the flow of pedestrians across the road
	\item A push button for pedestrians to signal that they want to cross
\end{itemize}
The safety constraints are:
\begin{itemize}
	\item Traffic and pedestrians should not be permitted in the crossing at
		the same time
	\item When a pedestrian signals that they want to cross, that request must
		be satisfied
	\item Timing constrains (see \ref{time-constraint}) and light
		sequences (see \ref{light-sequence}) must be followed.
\end{itemize}

\subsection{Light sequences}\label{light-sequence}
The lights in the crossing follow prescribed sequences, which must be adhered
to.
\subsubsection{Traffic light}
The traffic lights have Red, Amber\todo[color=blue!10]{use the blue led},and
Green lights.  These show in a fixed sequence;
\begin{enumerate}
	\item Red only
	\item Red and Amber
	\item Green
	\item Amber
	\item Red
\end{enumerate}

\subsubsection{Pedestrian lights}
The lights signalling to the pedestrians consists of a Red and Green lights.
The sequence for these is
\begin{enumerate}
	\item Red
	\item Green
	\item Red
\end{enumerate}

\subsection{Timings}\label{time-constraint}
The light sequences are subject to time constraints between transitions. 
These apply to the traffic lights performing the Red$\rightarrow$Green and
Green$\rightarrow$Red parts of the sequence.  Shown in  figure \ref{light-timing-diagram}

\subsubsection{Red$\rightarrow$Green}
The ``Red and Amber'' part of the sequence should show for 3 seconds

\subsubsection{Green$\rightarrow$Red}
The ``Amber'' part of the sequence should show for 3 seconds

\begin{figure}\centering
	\begin{tikztimingtable}
		{Red}  & 3H     3H     3L {[dotted]3L} 3L 3L 3H\\
		{Amber}& 3L     3H     3L {[dotted]3L} 3L 3H 3L\\
		{Green}& 3L     3L     3H {[dotted]3H} 3H 3L 3L\\
		{}     & 3SN(b) 3SN(e) 3S     3S       3SN(g) 3SN(h)\\
	\extracode
	\vertlines[black!20]{3,6, 15,18}
	\begin{pgfonlayer}{background}
		\draw[|<->|] (b)--node[below]{3s}(e);
		\draw[|<->|] (g)--node[below]{3s}(h);
	\end{pgfonlayer}
	\end{tikztimingtable}
	\caption{Traffic light timing sequence}
	\label{light-timing-diagram}
\end{figure}




\end{document}

