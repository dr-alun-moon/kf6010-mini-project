%&pdflatex
\documentclass{article}
\usepackage[british]{babel}

\usepackage{hyperref}
\usepackage{natbib}
\usepackage{tcolorbox}
\usepackage{booktabs,siunitx}

\usepackage{tikz-timing}

\usepackage[right=2cm, vscale=0.8, includemp, marginparsep=5mm,
marginparwidth=1.8cm]{geometry}

\title{KF6010 Mini Project}
\author{Dr Alun Moon\thanks{\url{alun.moon@northumbria.ac.uk}} and
Dr David Kendall\thanks{\url{david.kendall@northumbria.ac.uk}}}
\begin{document}
\maketitle

\begin{abstract}
\end{abstract}

\section{Introduction}
	This mini project runs at the end of the first semester, and can be viewed
	as a formative exercise to get feedback ready for the final assessment at
	the end of the module.
	\begin{quote}
		Do treat this as seriously as if it were an assignment.
		Because this isn't assessed, we can give you much more help and
		feedback than we can with the final assignment.  

		\emph{Take this opportunity to build up your knowledge and confidence.}
	\end{quote}

\section{Scenario}
You are tasked with writing a \emph{robust} low power controller for a
pedestrian crossing.  The crossing has standard light sequences for the
traffic and pedestrians.  In addition to the light signals there is an audible
crossing tone.  The system should be event driven.

The crossing is subject to various legislation and standards set out by the
Department for Transport.  Behaviour of the traffic signals is principally
covered in the ``The Traffic Signs Regulations and General Directions''
\citep{tsrgd}.  The sequences and timings for use in pedestrian crossings are
given in ``The Design of Pedestrian Crossings'' \citep{dps}.  The
specifications for the audible signal are in ``Performance Specifications for
Audible Equipment for use at Pedestrian Crossings''\citep{psaepc}

\subsection{Features}
A pedestrian crossing has the following features:
\begin{itemize}
	\item Traffic lights to control traffic on the road
	\item Pedestrian lights to control the flow of pedestrians across the road
	\item A push button for pedestrians to signal that they want to cross
\end{itemize}
The safety constraints are:
\begin{itemize}
	\item Traffic and pedestrians should not be permitted in the crossing at
		the same time
	\item When a pedestrian signals that they want to cross, that request must
		be satisfied
	\item Timing constrains (see~\ref{time-constraint}) and light
		sequences (see~\ref{light-sequence}) must be followed.
\end{itemize}

\section{Requirements}
%%- \begin{description}
%%- 	\item[interrupts] should be used for the button presses
%%- 		\item[mutexex] must be used for exclusive access to the crossing
%%- 	\item[Semaphres] should be used for signalling
%%- \end{description}

\subsection{Light sequences}\label{light-sequence}
The lights in the crossing follow prescribed sequences, which must be adhered
to.
\paragraph{Traffic light}
The traffic lights have Red, Amber
\marginpar{use the blue led for the amber light}, and
Green lights.  These show in a fixed sequence \citep[Schedule 14, para 4.2]{tsrgd};
\begin{enumerate}
\setlength{\itemsep}{-3pt}
	\item Red only
	\item Red and Amber
	\item Green
	\item Amber
	\item Red
\end{enumerate}

\paragraph{Pedestrian lights}
The lights signalling to the pedestrians consists of a Red and Green lights.
The sequence for these is
\begin{enumerate}
\setlength{\itemsep}{-3pt}
	\item Red
	\item Green
	\item Red
\end{enumerate}

\subsection{Timings}\label{time-constraint}
The light sequences are subject to time constraints between transitions. 
The timings are summarised in table~\ref{timing-sequence} taken from \citep[Table 6,
page 19]{dps}

\begin{table}[h]
	\caption{Timing sequence for crossing in seconds}
	\label{timing-sequence}
	\centering
	\begin{tabular}{rllSSl}\toprule
		Phase & pedestrians & vehicles & \multicolumn{2}{c}{timings} \\
			  &				&		   & min & max \\\midrule
		1 & Red & Green & \\
		2 & Red & Amber & 3 && \emph{mandatory} \\
		3 & Red & Red   & 1&3\\
		4 & Green & Red & 4 & 9 & with audible tone\\
		5 & Red & Red   & 1 & 5 \\
		6 & Red & Red and Amber & 2 && \emph{mandatory} \\
		\bottomrule
	\end{tabular}
\end{table}

\paragraph{Red and Amber}
The ``Red and Amber'' lights must show together for 2 seconds \citep[Schedule 14, para 4.3]{tsrgd}

\paragraph{Amber}
The single ``Amber'' light must show for 3 seconds \citep[Schedule 14, para 4.4]{tsrgd}

\section{Audible signal}
The green light indicating pedestrians may cross is accompanied by an audible
signal.  This has a frequency between 2kHz and 3.5kHz, pulsed at 240 pulses
per minute, with a mark space ratio of 1.5:1 \cite[para 2.5]{psaepc}


\begin{figure}
	\centering
\begin{tikztimingtable}
Red   & [red] 3H 2H 2L {[dotted]4L} 2L 3L 2H 5H 4H\\ 
Amber & [orange] 3L 2H 2L {[dotted]4L} 2L 3H 2L 5L 4L\\
Green & [green!80!black] 3L 2L 2H {[dotted]4H} 2H 3L 2L 5L 4L\\ 
	{} & \\
	Red   & [red] 7H {[dotted]4H} 2H 3H 2H 5L 4H \\
	Green & [green!80!black] 7L {[dotted]4L} 2L 3L 2L 5H 4L \\
	{} & \\
Button  & 7L {[dotted]4L} L H 14L\\
	{}    & 3SN(a)2SN(b) 7SN(c) 1SN(d)3SN(e)2SN(f) 5SN(g) 2SN(h)\\
\extracode
\vertlines[black!20]{3,5,13,16,18,23,25}
\vertlines[dashed,black!50]{12}
\begin{pgfonlayer}{background}
	\draw[<->] (a)--node[below]{2s}(b);
	\draw (c) node[pin=below left:interrupt]{};
	\draw[<->] (d)--node[below]{3s}(e);
	\draw[<->] (e)--node[below]{2s}(f);
	\draw[<->] (g)--node[below]{2s}(h);
\end{pgfonlayer}
\end{tikztimingtable}
\caption{Timing diagram}
\label{t}
\end{figure}

\section{Implementation}
\paragraph{A suggested solution may include}
\begin{itemize}
	\item Threads to handle 
		\begin{itemize}
			\item Traffic signals
			\item Pedestrian signals
		\end{itemize}
	\item Interrupt handlers to respond to the button pushes.
	\item Mutexes for safety critical code
	\item Semaphores for internal signalling and messaging
\end{itemize}

\paragraph{My sample solution has} \emph{3 threads, 2 interrupt handlers, 1
mutex, and 2 semaphores}


\section{Implementation notes}
More detailed notes on implementation issues will be released as an appendix
in the second week of the mini project.

\bibliographystyle{plainnat}
\bibliography{traffic-regs}

\end{document}

\clearpage\appendix
\section{Implementation}
\subsection{Frequency}
The tone can be generated with a PWM output.  The period for the PWM output is
calculated in table~\ref{tone-freq}

\begin{table}[h]
	\caption{Audible tone frequency}
	\label{tone-freq}
	\centering
\begin{tabular}{lSs}\toprule
	Frequency & 2 & \kilo\hertz \\
			  &  2000 & \hertz \\
	period	&	0.0005 & \second \\
			& 0.5	& \milli\second \\\midrule
PWM period	& 500 & \micro\second\\\bottomrule
\end{tabular}
\end{table}

\subsection{Pulse}
The pulse timing is calculated in tables~\ref{pulse-rate} and \ref{mark-space}
A mark-space ratio of 1.5:1 is 3:2\\

\begin{table}[h]
	\caption{Audible pulse timings}
	\label{pulse-rate}
	\centering
\begin{tabular}{lSs}\toprule
	Pulses & 240 & \per\minute \\
			& 4  & \hertz \\
	period & 0.25 & \second \\\midrule
		& 	250 & \milli\second\\\bottomrule
\end{tabular}
\end{table}


\begin{table}[h]
	\caption{Mark-space ratio}
	\label{mark-space}
	\centering
\begin{tabular}{lcSs}\toprule
	mark & $\frac{3}{5}$ period  & 150 & \milli\second \\
	space  & $\frac{2}{5}$ period  & 100 & \milli\second \\\bottomrule
\end{tabular}
\end{table}

\end{document}

